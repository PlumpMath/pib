\section{Crowd plug-in}

The goal of this work package was to implement the movement of the crowd. At first we wanted to implement a general movement algorithm and an other algorithm that would control the animation of the crowd. The second part was quite difficult to implement and thus we decided to concentrate ourselves on the first part.\\
Our algorithm is split into two parts one that is independant from Blender and one that depends on it. \\
The first part create a set of key points that will represent the movement of one person and the second interpolate a trajectory from those points.

% Il faudra changer les titres je pense 
\subsection{Algorithm description}

The first algorithm is inspired by those two papers: \cite{PLE} and \cite{vandenBerg2011}. We chose them because the notions involved in the describtion of the algorithm were more familiar to us.
% TODO : combler avec du bla bla
The idea is the following: a graph (grid) algorithm will generate a general path that for each individual and another algorithm will prevent collisions between individuals. We descretize the time in order to compute a next point for each individual easily. This is how we get the final set of points for each individuals.

\subsection{Algorithm implementation}

\subsubsection{The guide graph}

The first part that we tried to implement was the graph. We created a data structure reprensenting the nodes of the graph and the edges. This was quite easy to do since the graph is suposed to be a grid, it is regular.
Then we needed a minimum distance (1 to 1) algorithm on the graph. For that we choosed the $A^*$ algorithm. Unfortunaly, our implementation of the $A^*$ was realy costly in time and even more in memory. This fact forced us to abandon the graph in the rest of the devellopement. We could test it for small values, and for a small number of calls, but the $A^*$ would have to be called thousands of times which makes this impractical. 
To replace the graph we used the euclidian distance, this involved removing statics obstacles. Also the points were no more able to avoid packed places and just went strait to there goal.

\subsubsection{Allowed velocity field}

This part deals with collision avoidance. This part is the part that took us the more time to implement. The problem is the following, you have a set of individual (i.e. points) with current velocities. You want for each individuals a set of velocities that will ensure that  if we peek one in it then we will not colide with another individual on the way. To do that the algorithm makes a lot of geometrical computation as you can see in \cite{vandenBerg2011}. We used the python library Shalely to reprensent geometrical structures. This library has some very usefull tools but some functionalities do not work very well with the flotants and the small errors that they provied. The errors linked to the flotant are one of the main reason it took us time. The other was that some geometrical forms were hard to reprensent and to compute both theoriticaly and computationaly.
Also due to the absence of the guide graph, the individuals were not able to go around obstables so it induced bugs, the points tends to get closer and closer to the limits of the obstacle until they cross it through errors and thus go through obstacles.
In the end, this part does return collisions free velocities in 95\% of cases. There are still some bugs that we were not able to find.

\subsubsection{Computation of the allowed movement}

This part involves minimizing a function on the velocity field of each individuals. For that we had two options. The first one was to implement a simplex but the graph was not a linear constraint so we though that this was not a valid way to minimize. The second method was to choose an angle and incrementing it according to a $d\theta$ and minimizing the function on those angles.
This part is fully functionnal but relied on flotants again involving computional errors.

\subsection{Path generation in Blender}

An algorithm described in the previous paragraphs outputs individual's coordinates at every time step and from these we had to interpolate a continuous path in Blender. 

\subsubsection{Human animation}

\par At the beginning we considered animating Humans and started by analysing a theoretical survey of Computer animation of Human walking (\cite{th_walking}) and looking for what was already done in Blender concerning automatic walking. All Blender-related resources on walking animation are gathered on a web-page (\cite{blwikiwalking}). Walk-o-matic and stride add-ons were used in previous versions of Blender to ``help to interactively design rough passes of a walk'' and ``quickly create cycles for background or extra characters'', however both were broken on Blender 2.7 and we are using Blender 2.76. 

\par Due to the lack of existing tools concerning automatic walking we consider creating such a tool ourselves. We started by analysing tutorials on blender character animation (for examle \cite{tuto_walk}) and creating such a motion manually. However we soon realysed how difficult it is to generate a realistic walk due to the complex physics behind the movements and decided that it is out of the scope of our project and we have shifted our attention to the more basic task of creating a path in Blender and making an object following it with varying speed given by our algorithm

\subsection{Creating a path}

\par There are two ways to make an object move in Blender
\begin{enumerate}
\item Fix where an object should be at a given time and then modify the interpolation of the movement to make it realistic.  
\item Use Blender structures for paths (Bezier curves, NURBS curves) and various ways to couple an object and a path (Follow Path Constraint, Clamp To Constraint).
\end{enumerate} 

All of these options generate movement, however our work was to find the one that could be automated easily, would be compatible with a data structure given by our algorithm, would be precise and would be easy to modify for the user afterwards. 

The first option was compatible with our data structure however realistic interpolation of the path and possibility to modify a path afterwards posed us a lot of questions and and we have chosen the second option which is a more conventional one, makes a clear distinction between a path and an object following it and is easy to modify for an artist afterwards. 

The main problem to solve was a discrepancy between the data structures: position of an object on the path (Bezier or NURBS) is given through its distance along the path from the starting point while in the data structure given by our algorithm position of the point is accessed through its coordinates. This being said we had to find a way to link a position in the space to the length of the path from the starting point to that position. Blender 2.76 having no add-on to mesure the length of the path made this task more complicated as we had to familiarize with the mathematics behind the interpolation of Bezier curves and NURBS.   

We have chosen Bezier curves instead of NURBS because their mathematical properties allowed us to guarantee that an individual will be at a given place at a given time while NURBS interpolates a path that only comes close to the given points but not necessary passes through them.

\subsection{GUI}
The approach we chose was to have a GUI that would be integrated in the Blender GUI.\\
There where many layers of GUI in Blender:
\begin{itemize}
\item window
\item banner
\item pannel
\end{itemize}

We chose to make pannels for simplicity reasons and we put those in the right banner of the 3DVIEW window.\\
Once we mastered the making and integrating of pannels with simple functionnalities in Blender we started to separate the functionnalities we wanted for the GUI of the crowd plug-in.

\subsubsection{The Map Pannel}
The first pannel we decided to create was one to create the map and grid on which the crowd will evolve.\\
We created it with four main functionnalities:
\begin{itemize}
\item save and load the current map
\item adjust size and origin of the map
\item adjust grid size (crucial for the algorithm)
\item add exclusion zones
\end{itemize}


\subsubsection{The Parameters Pannel}
The second pannel provides three of methods to create a crowd 
\begin{itemize}
\item save and load
\item initialisation with default settings
\item initialisation with random settings
\end{itemize}

We also chose to add a personnalisation feature: from an existing crowd (initialized with one of the three methods above) the user can select an individual and change all of its settings: initial position, goal, size, optimal speed, maximal speed and animation of the individual.


\subsubsection{The Simulation Pannel}
The last pannel allows the user to launch the computation of a crowd animation and to render it in Blender.\\
The user must first set the time quantum, number of time quantum to be computed and angle quantum and then the user can launch computation and load the resulting animation into Blender's 3DVIEW.\\
There is also a save and load functionality for the animation.

\begin{figure}
\centering
\subfloat{\label{fig:first}\includegraphics<2>[height=6cm]{img/GUI_map_example.png}} ~ 
\subfloat{\label{fig:second}\includegraphics<2>[height=6cm]{img/GUI_crowd_example.png}} ~ 
\subfloat{\label{fig:third}\includegraphics<2>[height=6cm]{img/GUI_simulation_example.png}}
\label{3figs}

\end{figure}
