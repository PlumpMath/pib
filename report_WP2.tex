\documentclass[a4paper,11pt]{article}

\usepackage{geometry}
\geometry{hmargin=3cm,vmargin=3.5cm}

\usepackage[T1]{fontenc}
\usepackage[utf8]{inputenc}
\usepackage{lmodern}
\usepackage[english]{babel}

\usepackage{graphicx}
\usepackage{hyperref}
\usepackage{color}

\usepackage{appendix}

\usepackage{tikz}

%\usetikzlibrary{decorations.pathmorphing}
\usepgflibrary{decorations.pathreplacing}

\usepackage{comment,amsmath,amsthm,amssymb}
\usepackage[noend,linesnumbered]{algorithm2e}

\usepackage[shortlabels]{enumitem} %change the numbers in enumerate

\theoremstyle{definition}
  \newtheorem{definition}{Definition}[section] 
\theoremstyle{definition}
  \newtheorem*{notation}{Notation}
\theoremstyle{remark}
  \newtheorem*{remark}{Remark}
\theoremstyle{plain}
  \newtheorem{theorem}{Theorem}
\theoremstyle{lemma}
  \newtheorem{lemma}[theorem]{Lemma}
\theoremstyle{corollary}
  \newtheorem{corollary}[theorem]{Corollary}
\theoremstyle{definition}
  \newtheorem{ex}{Example}
  
%\renewcommand{\epsilon}{\varepsilon}
  
\begin{document}
  

\title{WP2 - Crowd bibliography}
\author{}
\date{}

%\maketitle 

The goal of this workpackage is to find how to simulation large crowds in a rather realistic manner. After a few researches we have noticed that the crowds can be divided into two types: moving crowds and motionless crowd. We have decided focus on moving crowds. 

There are two major difficulities in moving crowds. Firstly we have to generate the path that each person will follow, and then we have to coordination the motions of the character that is moving (its foot, its arms) in order to have a realistic crowd.

\paragraph{Generation of the path}~

The strategy used for the crowd simulation is the least-effort approach given in .%TODO bibliography.
 In this algorithm the trajectory is optimized by computing the minimum energy cost in between the starting and the arrival. To find this minimum we use Euler's method: we compute each time the minimum on a small distance and we approximate the whole trajectory with the computation of the smallest distance on a graph. This provides us a set of points that describe the path followed by each character.

This algorithm takes account of the velocity of each character. And the velocities are chosen so that there is no collision between two characters, and between a charactere and the environment.

The reachable points on the environnement are described by a level map, and by a set of exclusion polygones. A graph is used, as we said before, to approximate the remaining distance. So its edges are weighted by the distances between two points. Each character can go to different places. So we decided to allocate to each character a set of interest points with a given probability to go to them. For the sake of simplicity, we will start with only one goal per charactere.

Moreover we have thought about how to take count of the altitude. For example, if there is a montagne, it is sometime more efficiant to bypass it. The idea we had for that is to modify the weight of the edges in function of the potential energy. 

\paragraph{Motions of the characters}~

The first studies were based on the article .%TODO bibliography


\end{document}
