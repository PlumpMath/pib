\documentclass[a4paper,12pt]{article}
\usepackage[T1]{fontenc}
\usepackage[utf8]{inputenc}
\usepackage{lmodern}
\usepackage{url}
\usepackage[francais]{babel}

\usepackage{hyperref}

\title{Blend'it}
\author{Etienne Moutot, Raphaël Monat}

\begin{document}

\maketitle

\section{Membres du projet}
  \begin{itemize}
    \item Benjamin Boisson
    \item Guillaume Combette
    \item Dimitri Lajou
    \item Victor Lutfalla
    \item Octave Mariotti
    \item Raphaël Monat (Vice-coordinateur)
    \item Etienne Moutot (Coordinateur)
    \item Johanna Seif
    \item Pijus Simonaitis
  \end{itemize}

\section{Description}
Blend'it est un projet dont le but principal est d'apporter de nouvelles fonctionnalités au logiciel 3D open source Blender\cite{Blender}. Ce logiciel est un logiciel permettant de créer du début à la fin des scènes en images de synthèse: modélisation, texturing, animation et rendu. Il dispose même de quelques fonctionnalités de rendu temps-réel type jeu vidéo.

Le projet s'est tourné vers la génération automatique et/ou guidée de grands espaces peuplés. Cela consiste à générer les espace (par exemple des paysages naturels ou urbain, comme des forets ou des villes), et ensuite y ajouter une foule humaine, puis l'animer en fonction de son environnement.

Les recherches en informatiques graphique autour de la génération procédurale d'environnements, la génération et  l'animation de foules sont nombreuses et fournissent de nombreuse approches différentes pour développer notre outil. Le choix a été fait de développer une \textit{add-on} de Blender, développée en Python, qui utilisera les techniques citées plus haut pour fonctionner. Cela prendra la forme de deux add-on différentes, le plus indépendantes possibles, mais interfaçables : une pour générer un environnement, l'autre pour peupler cet environnement.

Les techniques de peuplement de scènes virtuelles existent depuis un moment, et sont pour la plupart disponible dans des logiciels de 3D commerciaux. La société ILM, avec le logiciel MASSIVE\cite{Massive} a révolutionné ce développement en interne pour créer les armées du \textit{Seigneur des Anneaux}. Un concurrent de MASSIVE est Goalem\cite{Golaem}. Aujourd'hui, même le logiciel généraliste 3DS MAX\cite{3dsmax} propose des fonctionnalités d'animation de foules réalistes. \\
Concernant Blender, et même en open source de manière générale, rien n'existe vraiment pour le moment. Un ancien script permettait de générer des foules de taille moyenne en situations de guerre, mais il donnait des résultats assez peu convaincants, et n'est surtout plus maintenu, et donc plus compatible avec les versions actuelles de Blender.

Le but est de permettre à Blender de combler son retard sur les logiciels propriétaires concurrents. Actuellement l'intégralité des studio professionnels fonctionnent avec des logiciels propriétaires (et souvent hors de prix). Faire avancer Blender à leur niveau sur une fonctionnalité précise, voire les dépasser, permettrait aux artistes d'avoir accès à un outil à la pointe de la recherche, gratuitement (car open source).

Le public visé forme actuellement une communauté déjà existante : les infographistes utilisant Blender pour créer leurs scènes 3D,. On peut même envisager d'attirer vers Blender des artistes utilisant pour le moment les logiciels 3D propriétaires, faute d'alternative open source.

\section{Calendrier prévisionnel}
\begin{description}
  \item[- 5 Nov.] (4 semaines):  \textit{Recherches prévisionnelles}. Collecte d'article et éventuellement de projets sur lesquels baser notre code. Exploration de plusieurs pistes quand aux méthodes à implémenter.
  \item[5 Nov. - 19 Nov.] (2 semaines): \textit{Recherches approfondies}. Une fois les pistes définitives choisies, approfondissement de celles-ci. Éventuellement début de pseudo-code pour expliciter les algorithmes utilisés.
  \item[19 Nov. - 3 Déc.] (2 semaines): \textit{Code de base et prise en main de Blender}. Codage des fonctions "de base" en python, ne nécessitant pas forcément Blender. En parallèle, apprentissage du code pour Blender avec les 3 "codeurs".
  \item[3 Dec. - 7 Jan.] (5 semaines): \textit{Implémentation}. Implémentation de tout dans Blender.
  \item[7 Jan. - 4 Fev.] (4 semaines) \textit{Interface graphique}. Peaufinent de l'interface graphique, éventuellement refonte et réflexion autour de l'ergonomie. Cette période est volontairement longue, parce que l'ergonomie sera un aspect primordial du logiciel, et que cette période sera aussi en partie celle de révision des partiels du semestre 1.
  \item[4 Fev. -] (en fonction de la soutenance..) \textit{Debug et ajout de fonctionnalités}.
  En fonction de la quantité de bugs et de temps qu'il nous reste, on peut envisage d'explorer quelques pistes laissées de côté dans les premières semaine, et enrichir de nouvelles fonctionnalités notre outil.
\end{description}

\subsection*{Deadlines prévues}
\begin{description}
  \item[5 Nov.] Décision des méthodes à implémenter. 
  \item[7 Jan.] Version fonctionnelle.
  \item[4 Fev.] Version utilisable par le public.
\end{description}


\section{Répartition du travail (au démarrage)}

Pour le début du projet, nous avons décidé de créer quatre pôles de recherche :
\begin{itemize}
\item Une équipe va explorer le code de Blender, son fonctionnement interne, ainsi que l'utilisation de l'API Python pour pouvoir la présenter ensuite à tout le groupe. Etienne, Pijus et Raphaël s'occupent de cette partie. 
\item Une équipe est chargée de rechercher ce qui a déjà été fait et publié concernant la génération ainsi que l'animation des foules. Dimitri, Johanna et Victor composent cette équipe.
\item Une équipe est chargée de rechercher ce qui a déjà été fait et publié concernant la génération de villes et d'architectures. Cette équipe est composée de deux membres : Benjamin et Octave.
\item Une équipe est chargée de recherche bibliographique concernant la génération de terrains "naturels". Guillaume est seul sur cette tâche.

Une fois le travail de recherche fini, les équipes se réorganiseront en fonction de ce qui aura été décidé d'implémenter. Les "codeurs" aideront le reste du groupe à comprendre comment coder pour Blender
\end{itemize}

\newpage

\begin{thebibliography}{}

\bibitem{Blender}
  Blender,
  \url{http://www.blender.org/}


\bibitem{Golaem}
  Golaem Crowd Simulation for Maya,
  \url{http://golaem.com/}


\bibitem{Massive}
  Massive Software,
  \url{http://www.massivesoftware.com/}


\bibitem{3dsmax}
  3DS MAX,
  \url{http://www.autodesk.fr/products/3ds-max/overview}

\end{thebibliography}


\end{document}
